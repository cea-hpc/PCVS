%%%%%%%%%%%%%%%%%%%%%%%%%%%%%%%%%%%%%%%%%%%%%%%%%%%%%%%%%%%%%%%%%%%%%%%%%%%%%
 %                                                                          %
 % Tue Jul 22 13:28:10 CEST 2014                                            %
 % Copyright or (C) or Copr. Commissariat a l'Energie Atomique              %
 %                                                                          %
 % This file is part of JCHRONOSS                                           %
 %                                                                          %
 % JCHRONOSS is free software: you can redistribute it and/or modify        %
 % it under the terms of the GNU Lesser General Public License as           %
 % published by the Free Software Foundation, either version 3 of the       %
 % License, or (at your option) any later version.                          %
 %                                                                          %
 % JCHRONOSS is distributed in the hope that it will be useful,             %
 % but WITHOUT ANY WARRANTY; without even the implied warranty of           %
 % MERCHANTABILITY or FITNESS FOR A PARTICULAR PURPOSE.  See the            %
 % GNU Lesser General Public License for more details.                      %
 %                                                                          %
 % You should have received a copy of the GNU Lesser General Public License %
 % along with JCHRONOSS.  If not, see <http://www.gnu.org/licenses/>.       %
 %                                                                          %
 % Version : 1.0                                                            %
 % Author  : Julien Adam <julien.adam@cea.fr>                               %
 %                                                                          %
 %%%%%%%%%%%%%%%%%%%%%%%%%%%%%%%%%%%%%%%%%%%%%%%%%%%%%%%%%%%%%%%%%%%%%%%%%%%%

\documentclass[10pt]{article}
\usepackage[T1]{fontenc}
\usepackage[utf8]{inputenc} %encoding

\title{\textbf{JCHRONOSS} \\ Jobs runner with integrated optimized tasks scheduler in High Performance contexts}
\author{Julien ADAM \and Marc PÉRACHE}
\date{\today}

\begin{document}
\maketitle
\thispagestyle{empty} % remove page number
A trend in High Performance Computing (HPC) is the increasing growth of computing power, thanks to modern and future technologies. Efficient using of these parallel architectures is a real challenge and major stake in a lot of fields as scientific simulations. Thus, some research projects have been created with the objective of handling this new technology by taking innovative concepts. In well-structured research project, a necessary step consists of validating, in order to ensure the product conformity with expected results. But, in HPC context, this way is harder, due to specifics constraints and complex architectures. Standards solutions have troubles to solve these issues, in particularly with scalability questions. Our solution presented here, is a jobs runner following quality checker requirements, but also able to interact cleverly with high computing models. Our approach is based on the fact that the application is aware about its environment and especially about HPC specifications. Reflecting of this, we can evaluate computers needs and divide efficiently the workload. This way, it is possible to improve significantly validation step yields, whose ensue resources and time gains. The high software compatibility with most of jobs managers currently used on supercomputers is a benefit and increases its scope. Thanks to several algorithmic policies, \textit{JCHRONOSS} can adjust its behaviour about the system. This method allows it to increase its flexibility. Several functionalities working around these most important concepts with the objective of make easier \textit{JCHRONOSS} integration with more complex mechanisms. Finally, its eases of installing, configuring and using make it a jobs runner with a wide scope. In summary, \textit{JCHRONOSS} purpose is to give high--programming projects the opportunity to be in line with agile software development methods, these one still remaining today, pledges of software quality. 
\end{document}
